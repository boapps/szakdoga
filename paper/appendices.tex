\appendix

\chapter*{Függelék}\addcontentsline{toc}{chapter}{Függelék}
\setcounter{chapter}{6}  % a fofejezet-szamlalo az angol ABC 6. betuje (F) lesz
\setcounter{equation}{0} % a fofejezet-szamlalo az angol ABC 6. betuje (F) lesz
\numberwithin{equation}{section}
\numberwithin{figure}{section}
\numberwithin{lstlisting}{section}
%\numberwithin{tabular}{section}

\section{K-Monitor módszertana}
\label{appendix:kmonitor-methodology}

\emph{A következő egy minimálisan módosított kivonat a módszertanukból}:


Új cikk felvételének az alábbi kritériumai vannak:

\begin{enumerate}
	\item konkrét korrupciós esetet ír le
	\item cikk szerzője állítja/sugallja, hogy valaki a ráruházott hatalmat saját maga, vagy egy harmadik fél hasznára fordította
	\item cikk korrupciós ügyben történő jogi eljárásról tájékoztat
	\item korrupciós vádat cáfolnak vagy védekeznek
	\item következő témák esetében szabálytalanságok merülnek fel: közbeszerzés, pártfinanszírozás, pályázatok, kormányzat szerv vagy állami vállalat gazdálkodása, vagyonosodás, juttatások, privatizáció, whistleblowing
	\item következő kifejezések közül valamelyik előfordul a cikkben: korrupció, sikkasztás (közszolga által), hűtlen kezelés, vesztegetés, hivatali visszaélés, hatalommal való visszaélés, befolyással üzérkedés, hanyagság, adócsalás, számviteli fegyelem megsértése, protekció, nepotizmus, jogosulatlan gazdasági előny, versenykorlátozás, kartell, whistleblowing/közérdekű bejelentés, közérdekű adatok, átláthatóság
\end{enumerate}

Felvételt kizáró okok:
\begin{itemize}
	\item pártközlemény
	\item publicisztika
	\item mocskolódás
	\item nem saját anyag, más lapra hivatkozik és nem tartalmaz új információt
\end{itemize}

A cikkeket metaadatokkal is ellátták az alábbiak szerint:

\begin{itemize}
	\item Személyek: aki a cikk alapján negatív összefüggésbe hozható az eseménnyel, vagy korábban negatív összefüggésbe hozták, és tisztázták a vádak alól.
	\item Intézmények: a cikk szerint a cselekményben érintett cég, hivatal, szervezet (gazdaság, politika, társadalom) - tehát itt is csak a negatívan említettek!
	\item Helyek: ha mérvadó az ügy helyszíne (település/megye/ország). Csak akkor alkalmazzuk, ha az ügy szempontjából releváns és állami/önkormányzati szereplőről van szó.
	\item Egyéb kulcsszavak: az ügy témakörét adja meg, fontos jellemzőket rögzít. 
\end{itemize}

\clearpage\section{Használt promptok}

\subsection{Címkék generálása}

\begin{verbatim}
[címkék generálása]
{title}

{description}

{text}

cimkék: {keywords}

###

korrupciós címkék: {output}
\end{verbatim}

\subsection{Entitás klasszifikáció (személyek)}

\begin{verbatim}
[személy klasszifikáció]
{text}

###

összes: {all}
korrupcióban érintett: {output}
\end{verbatim}

\subsection{Entitás klasszifikáció (intézmények)}

\begin{verbatim}
[intézmény klasszifikáció]
{text}

###

összes: {all}
korrupcióban érintett: {output}
\end{verbatim}

\subsection{Reláció kinyerés}

\begin{verbatim}
Bekezdés:
"{text}"

Relációk:
{output}
\end{verbatim}

ahol az output az alábbi struktúrájú objektumokat tartalmazó tömb:

\begin{verbatim}
Indoklás: {explanation}
Kapcsolat: {relation}
Tárgy: {subject}
Alany: {object}
\end{verbatim}