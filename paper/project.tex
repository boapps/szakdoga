%--------------------------------------------------------------------------------------
% Feladatkiiras (a tanszeken atveheto, kinyomtatott valtozat)
%--------------------------------------------------------------------------------------
\clearpage
\begin{center}
\large
\textbf{FELADATKIÍRÁS}\\
\end{center}

A természetes nyelvű szövegek feldolgozása az informatika egy régi és egyre nagyobb fontossággal bíró területe. Az információkeresés és -kinyerés, az üzleti intelligencia rendszerek, a természetes nyelvű ember-gép interfészek, a közösségi média álhíreinek detektálása, vagy például történeti és irodalmi szövegek számítógépes analízise mind olyan területek, ahol ezeknek a módszereknek egyre nagyobb szerep jut.

A feladat egy olyan rendszer megvalósítása, amely az internetről származó rövid, természetes nyelvű szöveges hírekből felépített korpuszon képes információkeresési és -kinyerési feladatok megoldására. A rendszernek alkalmasnak kell lennie egy szövegbázis betöltésére (közben a releváns tartalmak kiválasztására), majd az így létrehozott korpuszon különféle elemzési és keresési feladatok megoldására, pl. entitásfelismerés, szemantikus annotálás, relevancialapú dokumentum-részletek keresése, információkinyerés, kérdés-megválaszolás stb. A feladat megoldása során azt is meg kell vizsgálni, hogy az elmúlt években megjelent nagy nyelvi modellek (large language models) alkalmazásának milyen lehetőségei vannak ezen a területen.

A hallgató munkájának a következőkre kell kiterjednie:
\begin{itemize}
	\item a feladatkiíráshoz hasonló célú megoldások felkutatása és megismerése,
	\item a rendszer fejlesztésére és tesztelésére alkalmas dokumentumgyűjtemény létrehozása,
	\item a fentebb megfogalmazott feladatokat megoldó alkalmazás tervezése és megvalósítása,
	\item a nagy nyelvi modellek lehetséges alkalmazásainak vizsgálata,
	\item teljesítménymértékek meghatározása az alkalmazás működésének kiértékelésére,
	\item információkeresési kísérletek futtatása, az alkalmazás működésének értékelése.
\end{itemize}



